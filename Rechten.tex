\documentclass[a1paper,landscape,25pt]{tikzposter}

\usepackage{amsmath}
\usepackage{pgf,tikz,pgfplots}
\pgfplotsset{compat=1.15}
\usepackage{mathrsfs}
\usetikzlibrary{arrows}


\settitle{
  \vspace{-5cm}
  \centering \vbox{
\@titlegraphic \\[\TP@titlegraphictotitledistance] \centering
\color{titlefgcolor} {\bfseries \fontsize{90}{100} \sc \@title \par}
\vspace{0em}
}}

\title{Rechten}

\addtolength{\jot}{1em}

\usetheme{Basic}
\usecolorstyle[colorOne=teal,colorTwo=cyan,colorThree=gray]{Spain}

\tikzposterlatexaffectionproofoff

\definenotestyle{notestyle}{
  targetoffsetx=0pt, targetoffsety=0pt, angle=90, radius=3cm,
  width=6cm, connection=false, rotate=0, roundedcorners=20, linewidth=1pt,
  innersep=1cm
}{
  \ifNoteHasConnection
  \draw[thick] (notecenter) -- (notetarget) node{$\bullet$};
  \fi
  \draw[draw=notebgcolor,fill=notebgcolor,rotate=\noterotate,rounded corners=\noteroundedcorners]
  (notecenter.south west) rectangle (notecenter.north east);
}
\usenotestyle{notestyle}

\begin{document}

\maketitle

\begin{columns}
  \column{0.7}
  \block{Grafisch}{
    \vspace*{3.5cm}
    \begin{center}
      \begin{tikzpicture}[scale=1.8,line cap=round,line join=round,>=triangle 45,x=1.0cm,y=1.0cm]
        \begin{axis}[
          x=2.0cm,y=2.0cm,
          axis lines=middle,
          xmin=-7, xmax=7, ymin=-1.5, ymax=5.5,
          ticks=none,
          ]
          \clip(-7,-1.5) rectangle (7,5.5);
          \draw [line width=2.6pt,domain=-7:7]
          plot(\x,{(--4--\x)/2});
          \draw [line width=2.pt,dotted] (2,3)-- (4,3);
          \draw [line width=2.pt,dotted] (4,3)-- (4,4);
          \draw (-3.4,0) node[anchor=north] {$(-\frac{q}{m},0)$};
          \draw (0,2.2) node[anchor=east] {$(0,q)$};
          \draw (2.1,2.9) node[anchor=south east] {\small$(x_1,y_1)$};
          \draw (4.1,3.9) node[anchor=south east] {\small $(x_2,y_2)$};
          \draw (3,3) node[anchor=north] {$\Delta x$};
          \draw (4,3.5) node[anchor=west] {$\Delta y$};
          \draw (2.5,3.2) node[anchor=west] {\small $\theta$};
          \draw (-6,4) node[anchor=north west] {\LARGE $y=mx+q$};
          \draw [fill=black] (-4,0) circle (2.5pt);
          \draw [fill=black] (0,2) circle (2.5pt);
          \draw [fill=black] (2,3) circle (2.5pt);
          \draw [fill=black] (4,3) circle (2.5pt);

          \draw [fill=black] (2,0) circle (2.5pt);
          \draw [fill=black] (4,0) circle (2.5pt);
          \draw (2,0) node[anchor=north] {\small$x_1$};
          \draw (4,0) node[anchor=north] {\small $x_2$};
        \end{axis}
      \end{tikzpicture}
    \end{center}
    \vspace*{3.5cm}
  }
  \column{0.3}
  \block{Rico}{
    \vspace*{1cm}
    \LARGE
    \begin{align*}
      m &= \dfrac{y_2-y_1}{x_2-x_1}
    \end{align*}
      \vspace*{1cm}
  }
  \block{Differentiequotiënt}{
      \vspace*{1cm}
    \LARGE
    \begin{align*}
      \dfrac{\Delta y}{\Delta x}_{[x_1,x_2]} &= \dfrac{f(x_2)-f(x_1)}{x_2-x_1}
    \end{align*}
      \vspace*{1cm}
  }
  \block{Hellingshoek $\theta$}{
      \vspace*{1cm}
    \LARGE
    \begin{align*}
      \tan \theta &= \dfrac{\Delta y}{\Delta x}
    \end{align*}
      \vspace*{1cm}
  }

\end{columns}

  \begin{columns}
    \column{0.5}
    \block{Rico $m$ en één punt $(x_1,y_1)$}{
      \vspace*{1cm}
      \LARGE
      $$y - y_1 = m(x - x_1) \qquad \mbox{of} \qquad \begin{vmatrix}x & y & 1 \\ x_1 & y_1 & 1 \\ 1 & m & 0\end{vmatrix} = 0$$
      \vspace*{1cm}
    }
    \column{0.5}
    \block{Punten $(x_1,y_1)$ en $(x_2, y_2)$}{
      \vspace*{1cm}
      \LARGE
        $$y - y_1 = \dfrac{y_2-y_1}{x_2-x_1}(x - x_1) \qquad \mbox{of} \qquad \begin{vmatrix}x & y & 1 \\ x_1 & y_1 & 1 \\ x_2 & y_2 & 1\end{vmatrix} = 0$$
      \vspace*{1cm}
    }
    % \column{0.6}
    % \block{Rechten in de ruimte}{
    %   \LARGE
    %   \begin{minipage}{.5\linewidth}
    %     \innerblock{\Large Stelsel parametervergelijkingen}{
    %       $$\begin{cases}
    %         x &= x_0 + ka\\
    %         y &= y_0 + kb\\
    %         z &= z_0 + kc
    %       \end{cases}$$
    %     }
    %   \end{minipage}
    %   \begin{minipage}{.5\linewidth}
    %     \innerblock{\Large Stelsel cartesische vergelijkingen}{
    %       $$\dfrac{x-x_0}{a}=\dfrac{y-y_0}{b}=\dfrac{z-z_0}{c}$$
    %     }
    %   \end{minipage}
    % }
  \end{columns}

\end{document}
