\documentclass[a1paper,landscape,25pt]{tikzposter}

\usepackage{amsmath}
\usepackage{amsfonts}

\settitle{
  \vspace{-5cm}
  \centering \vbox{
\@titlegraphic \\[\TP@titlegraphictotitledistance] \centering
\color{titlefgcolor} {\bfseries \fontsize{90}{100} \sc \@title \par}
\vspace{0em}
}}

\title{Machten en Machtswortels}

\addtolength{\jot}{1em}

\usetheme{Basic}
\usecolorstyle[colorPalette=GreenGrayViolet]{Spain}

\tikzposterlatexaffectionproofoff

\definenotestyle{notestyle}{
  targetoffsetx=0pt, targetoffsety=0pt, angle=90, radius=3cm,
  width=6cm, connection=false, rotate=0, roundedcorners=20, linewidth=1pt,
  innersep=1cm
}{
  \ifNoteHasConnection
  \draw[thick] (notecenter) -- (notetarget) node{$\bullet$};
  \fi
  \draw[draw=notebgcolor,fill=notebgcolor,rotate=\noterotate,rounded corners=\noteroundedcorners]
  (notecenter.south west) rectangle (notecenter.north east);
}
\usenotestyle{notestyle}

\begin{document}

\maketitle

\begin{columns}
  \column{0.5}

    \block{Definitie machten}{
      \LARGE
      \begin{align*}
        a^n &= a\cdot a \cdot a \ldots \cdot a\\
        a^1 &= a \qquad
              a^0 = 1
      \end{align*}
    }
    \subcolumn{0.5}
    \block{Negatieve exponent}{
      \Large
      $$a^{-n} = \dfrac{1}{a^n}$$
    }

  \block{Rekenregels machten}{
    \LARGE
    \begin{align*}
      a^m\cdot a^n &= a^{m+n} \\%&& \mbox{\Large machten met zelfde grondtal vermenigvuldigen}\\
      \dfrac{a^m}{a^n} &= a^{m-n} \\%&& \mbox{\normalsize machten met zelfde grondtal delen}\\
      \left(a^m\right)^{^n} &= a^{m\cdot n} \\%&& \mbox{\normalsize macht van een macht}\\
      \left(ab\right)^m &= a^mb^m \\%&& \mbox{\normalsize macht van een product}\\
      \left(\dfrac{a}{b}\right)^m &= \dfrac{a^m}{b^m} %&& \mbox{\normalsize macht van een quotiënt}
    \end{align*}
  }

  \block{Rationale exponent}{
    \LARGE
    \centering
    $a^{n/m} = \sqrt[m]{a^n}$
  }

  \column{0.5}

  \begin{subcolumns}
    \subcolumn{0.5}
    \note[targetoffsetx=41.5cm, targetoffsety=41cm, width=8cm]{
      \small\centering
      $a\in\mathbb{R}^+, b\in\mathbb{R}^+_0$
    }
    \block{Rekenregels vierkantswortels}{
      \vspace{0.4cm}
      \LARGE
      \begin{align*}
        \sqrt{ab} &= \sqrt{a}\sqrt{b}\\
        \sqrt{\dfrac{a}{b}} &= \dfrac{\sqrt{a}}{\sqrt{b}}\\
        \sqrt{a^m} &= \sqrt{a}^{^m}
      \end{align*}
    }

    \block{}{
      \vspace{-2cm}
      \LARGE
      $$\sqrt{a^2} = |a|$$
    }

    \block{}{
      \vspace{-2cm}
      \LARGE
      $$\sqrt{a} = \sqrt[2]{a}$$
    }

    \subcolumn{0.5}
    \block{Rekenregels machtswortels}{
      \LARGE
      \vspace{0.5cm}
      \begin{align*}
        \sqrt[n]{ab} &= \sqrt[n]{a}\sqrt[n]{b}\\
        \sqrt[n]{\dfrac{a}{b}} &= \dfrac{\sqrt[n]{a}}{\sqrt[n]{b}}\\
        \sqrt[n]{a^m} &= \sqrt[n]{a}^{^m}\\
        \sqrt[n]{\sqrt[m]{a}} &= \sqrt[m\cdot n]{a}\\
        \sqrt[n\cdot p]{a^{m\cdot p}} &= \sqrt[n]{a^m}\\
        \sqrt[n]{a^n} &= a
      \end{align*}
    }
  \end{subcolumns}

  \block{Definitie vierkantswortel \hfill $\pm\sqrt{a}$}{
    \vspace{0.5cm}
    \begin{center}
      \LARGE
      $x$ is een {\bf vierkantswortel} uit $a$ $\Leftrightarrow$ $x^2=a$\\[0.7cm]
    \end{center}
  }

  \block{Definitie machtswortel \hfill $\pm\sqrt[n]{a}$}{
    \vspace{0.5cm}
    \begin{center}
      \LARGE
      $x$ is een {\bf n-de machtswortel} uit $a$ $$\Updownarrow$$ $$x^n=a$$\\[0.7cm]
    \end{center}
  }

\end{columns}

\end{document}
