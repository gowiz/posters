\documentclass[a1paper,landscape,25pt]{tikzposter}

\usepackage{amsmath}

\settitle{
  \vspace{-5cm}
  \centering \vbox{
\@titlegraphic \\[\TP@titlegraphictotitledistance] \centering
\color{titlefgcolor} {\bfseries \fontsize{90}{100} \sc \@title \par}
\vspace{0em}
}}

\title{Machten en Machtswortels}

\addtolength{\jot}{1em}

\usetheme{Basic}
\usecolorstyle[colorPalette=GreenGrayViolet]{Spain}

\tikzposterlatexaffectionproofoff

\definenotestyle{notestyle}{
  targetoffsetx=0pt, targetoffsety=0pt, angle=90, radius=3cm,
  width=6cm, connection=false, rotate=0, roundedcorners=20, linewidth=1pt,
  innersep=1cm
}{
  \ifNoteHasConnection
  \draw[thick] (notecenter) -- (notetarget) node{$\bullet$};
  \fi
  \draw[draw=notebgcolor,fill=notebgcolor,rotate=\noterotate,rounded corners=\noteroundedcorners]
  (notecenter.south west) rectangle (notecenter.north east);
}
\usenotestyle{notestyle}

\begin{document}

\maketitle

\begin{columns}
  \column{0.5}

  \begin{subcolumns}
    \subcolumn{0.5}
    \block{Definitie machten}{
      \Large
      \begin{align*}
        a^n &= a\cdot a \cdot a \ldots \cdot a\\
        a^1 &= a \qquad
              a^0 = 1
      \end{align*}
    }
    \subcolumn{0.5}
    \block{Negatieve exponent}{
      \Large
      $$a^{-n} = \dfrac{1}{a^n}$$
    }
  \end{subcolumns}

  \block{Rekenregels machten}{
    \Large
    \begin{align*}
      a^m\cdot a^n &= a^{m+n} && \mbox{\normalsize machten met zelfde grondtal vermenigvuldigen}\\
      \dfrac{a^m}{a^n} &= a^{m-n} && \mbox{\normalsize machten met zelfde grondtal delen}\\
      \left(a^m\right)^{^n} &= a^{m\cdot n} && \mbox{\normalsize macht van een macht}\\
      \left(ab\right)^m &= a^mb^m && \mbox{\normalsize macht van een product}\\
      \left(\dfrac{a}{b}\right)^m &= \dfrac{a^m}{b^m} && \mbox{\normalsize macht van een quotiënt}\\
    \end{align*}
  }
  \note[targetoffsetx=17cm, targetoffsety=-3cm, width=15cm]{
    \Huge
    $$ a^{n/m} = \sqrt[m]{a^n}$$
  }
  \block{Definitie vierkantswortel \hfill $\pm\sqrt{a}$}{
    \begin{center}
      \Large
      $x$ is een {\bf vierkantswortel} uit $a$ $\Leftrightarrow$ $x^2=a$\\
    \end{center}
  }

  \block{Definitie machtswortel \hfill $\pm\sqrt[n]{a}$}{
    \begin{center}
      \Large
      $x$ is een {\bf n-de machtswortel} uit $a$ $$\Updownarrow$$ $$x^n=a$$
    \end{center}
  }

  \column{0.5}

  \begin{subcolumns}
    \subcolumn{0.5}
    \block{Rekenregels vierkantswortels}{
      \vspace{-1cm}
      \begin{align*}
        \sqrt{ab} &= \sqrt{a}\sqrt{b}\\
        \sqrt{\dfrac{a}{b}} &= \dfrac{\sqrt{a}}{\sqrt{b}}\\
        \sqrt{a^m} &= \sqrt{a}^{^m}\\
        \sqrt{a^2} &= a
      \end{align*}
    }

    \block{}{
      \vspace{-1cm}
      $$\sqrt{a} = \sqrt[2]{a}$$
    }

    \subcolumn{0.5}
    \block{Rekenregels machtswortels}{
      \begin{align*}
        \sqrt[n]{ab} &= \sqrt[n]{a}\sqrt[n]{b}\\
        \sqrt[n]{\dfrac{a}{b}} &= \dfrac{\sqrt[n]{a}}{\sqrt[n]{b}}\\
        \sqrt[n]{a^m} &= \sqrt[n]{a}^{^m}\\
        \sqrt[n]{\sqrt[m]{a}} &= \sqrt[m\cdot n]{a}\\
        \sqrt[n\cdot p]{a^{m\cdot p}} &= \sqrt[n]{a^m}\\
        \sqrt[n]{a^n} &= a
      \end{align*}
    }
  \end{subcolumns}

  \block{Lijst met gehele machten}{
    \Large
    \begin{center}
      \begin{tabular}{|c|c|c|c|c|c|c|c|c|c|}
        \hline
        $a^n$ & $2$ & $3$ & $4$ & $5$ & $6$ & $7$ & $8$ & $9$ & $10$\\
        \hline
        2     & 4   & 8   & 16  & 32  & 64  & 128 & 256 & 512 & 1024\\
        \cline{1-10}
        3     & 9   & 27  & 81  & 243 & 729\\
        \cline{1-6}
        4     & 16  & 64  & 256 & 1024\\
        \cline{1-5}
        5     & 25  & 125 & 625\\
        \cline{1-4}
        6     & 36  & 216\\
        \cline{1-3}
        7     & 49  & 343\\
        \cline{1-3}
        8     & 64  & 512\\
        \cline{1-3}
        9     & 81  & 729\\
        \cline{1-6}
        10    & 100  & 1\,000 & $10^4$ & $10^5$ & $10^6$\\
        \cline{1-6}
        11    & 121\\
        \cline{1-2}
        12    & 144\\
        \cline{1-2}
        13    & 169\\
        \cline{1-2}
        14    & 196\\
        \cline{1-2}
        15    & 225\\
        \cline{1-2}
      \end{tabular}
    \end{center}

  }
\end{columns}

\end{document}
