\documentclass[a1paper,landscape,25pt]{tikzposter}

\usepackage{amsmath}
\usepackage{tikz}
\usetikzlibrary{calc}
\usepackage[thicklines]{cancel}
\usepackage{makecell}
\usepackage{multicol}
\usepackage{tabu}

\newcommand{\tikzmark}[2]{\tikz[baseline,anchor=base,remember picture] \node (#1) {#2};}

\usepackage{graphicx}
\usepackage{calc}
\newlength{\depthofsumsign}
\setlength{\depthofsumsign}{\depthof{$\sum$}}
\newlength{\totalheightofsumsign}
\newlength{\heightanddepthofargument}

\newcommand{\nsum}[1][1.4]{% only for \displaystyle
    \mathop{%
        \raisebox
            {-#1\depthofsumsign+1\depthofsumsign}
            {\scalebox
                {#1}
                {$\displaystyle\sum$}%
            }
    }
}
\newcommand{\resum}[1]{%
    \def\s{#1}
    \mathop{
        \mathpalette\resumaux{#1}
    }
}

\newcommand{\resumaux}[2]{% internally
    \sbox0{$#1#2$}
    \sbox1{$#1\sum$}
    \setlength{\heightanddepthofargument}{\wd0+\dp0}
    \setlength{\totalheightofsumsign}{\wd1+\dp1}
    \def\quot{\DivideLengths{\heightanddepthofargument}{\totalheightofsumsign}}
    \nsum[\quot]%
}

% http://tex.stackexchange.com/a/6424/16595
\makeatletter
\newcommand*{\DivideLengths}[2]{%
  \strip@pt\dimexpr\number\numexpr\number\dimexpr#1\relax*65536/\number\dimexpr#2\relax\relax sp\relax
}
\makeatother

%\makeatletter
%\newcommand\HUGE{\@setfontsize\Large{60}{100}}
%\makeatother

\settitle{
  \vspace{-5cm}
  \centering \vbox{
\@titlegraphic \\[\TP@titlegraphictotitledistance] \centering
\color{titlefgcolor} {\bfseries \fontsize{90}{100} \sc \@title \par}
\vspace{0em}
}}

\title{Combinatieleer}

\addtolength{\jot}{1em}

\usetheme{Basic}
\usecolorstyle[colorPalette=BrownBlueOrange]{Spain}

\tikzposterlatexaffectionproofoff

\definenotestyle{notestyle}{
  targetoffsetx=0pt, targetoffsety=0pt, angle=90, radius=3cm,
  width=6cm, connection=false, rotate=0, roundedcorners=20, linewidth=1pt,
  innersep=1cm
}{
  \ifNoteHasConnection
  \draw[thick] (notecenter) -- (notetarget) node{$\bullet$};
  \fi
  \draw[draw=notebgcolor,fill=notebgcolor,rotate=\noterotate,rounded corners=\noteroundedcorners]
  (notecenter.south west) rectangle (notecenter.north east);
}
\usenotestyle{notestyle}

\begin{document}

\maketitle

\begin{columns}
  \column{0.6}

  \block{Telproblemen}{
    \Large
    \begin{minipage}{0.3\linewidth}
      \vspace{1cm}
      $k$ : aantal beslissingen                                                           \\
      $n$ : aantal keuzes                                                                 \\
    \end{minipage}
    \begin{minipage}{0.5\linewidth}
      \begin{center}
        \tabulinesep2mm
        \begin{tabu}{c|c|c}
                             & \xcancel{Herhaling}                            & Herhaling \\
          \hline
          Volgorde           & \makecell{Variatie $V^k_n$ \\ Permutatie $P_n$} & Herhalingsvariatie $\bar{V}^k_n$  \\
          \hline
          \xcancel{Volgorde} & Combinatie $C^k_n$                             & Herhalingscombinatie $\bar{C}^k_n$
        \end{tabu}
      \end{center}
    \end{minipage}
  }

  \block{Formules}{
    \Huge
    \begin{align*}
      V^k_n & = {n! \over (n-k)!}          & \bar{V}^k_n & = n^k \\
      C^k_n & = {n! \over k! \cdot (n-k)!} & \bar{C}^k_n & = C^k_{n+k-1}
    \end{align*}
  }
  \note[targetoffsetx=-22cm, targetoffsety=-3cm, width=6cm]{\centering$P_n=n!$}
  \block{Faculteiten}{
    \vspace{-1em}
    \Huge
    \begin{align*}
      n!&=n\cdot(n-1)\cdot \ldots \cdot 2\cdot1\\
      0!&=1
    \end{align*}
  }
  \note[targetoffsetx=17cm, targetoffsety=-6.5cm, angle=90, radius=3cm,
  width=16cm, linewidth=.2cm,
  roundedcorners=20, innersep=1cm]{\large \centering $n!=\begin{cases}
      1        &\mbox{ als } n=0\\
      n \cdot (n-1)! &\mbox{ als } n>0
    \end{cases}$}
  \block{Anagrammen \large(Herhalingspermutaties)}{
    \huge
    \centering
    $\bar{P}^{^{n_1,n_2,\dots,n_k}}_n = \dfrac{n!}{n_1!\cdot n_2!\cdot \ldots \cdot n_k!}$
    \qquad \large of eleganter \qquad \Large
    $\bar{P}_{n_1,\dots,n_k} = \dfrac{\left(\sum_{i=1}^k n_i\right)!}{\prod_{i=1}^k n_i!}$
  }

  \column{0.4}

  \block{Variaties}{
    \large
    \begin{align*}
      V^k_n &= n \cdot (n-1) \cdot \ldots \cdot (n-k+1)\\
            &= \dfrac{n \cdot (n-1) \cdot \ldots \cdot (n-k+1) \, \color{gray}{ \cdot \, (n-k) \cdot \ldots \cdot 1}}{\color{gray}{(n-k) \cdot (n-k-1) \cdot \ldots \cdot 1}}\\
            &= \dfrac{n!}{(n-k)!}
    \end{align*}
  }

  \block{Driehoek van Pascal}{
    \Large
    \begin{center}
      \setlength{\tabcolsep}{9pt}
      \begin{tabular}{>{$}l<{$}|*{13}{c}}
        \multicolumn{1}{l}{$n$}                                                                     \\
        \cline{1-1}
        0                    & 1                                                                    \\
        1                    & 1 & 1                                                                \\
        2                    & 1 & 2  & 1                                                           \\
        3                    & 1 & 3  & 3  & 1                                                      \\
        4                    & 1 & 4  & 6  & 4   & 1                                                \\
        5                    & 1 & 5  & 10 & 10  & 5   & 1                                          \\
        6                    & 1 & 6  & 15 & 20  & 15  & 6   & 1                                    \\
        7                    & 1 & 7  & 21 & 35  & 35  & 21  & 7   & 1                              \\
        8                    & 1 & 8  & 28 & 56  & 70  & 56  & 28  & 8   & 1                        \\
        9                    & 1 & 9  & 36 & 84  & 126 & 126 & 84  & 36  & 9   & 1                  \\
        10                   & 1 & 10 & 45 & 120 & 210 & 252 & 210 & 120 & 45  & 10  & 1            \\
        11                   & 1 & 11 & 55 & 165 & 330 & 462 & 462 & 330 & 165 & 55  & 11 & 1       \\
        12                   & 1 & 12 & 66 & 220 & 495 & 792 & 924 & 792 & 495 & 220 & 66 & 12 & 1  \\
        \hline
        \multicolumn{1}{l}{} & 0 & 1  & 2  & 3   & 4   & 5   & 6   & 7   & 8   & 9   & 10 & 11 & 12 \\
        \cline{2-14}
        \multicolumn{1}{l}{} & \multicolumn{12}{c}{$k$}
      \end{tabular}
    \end{center}
  }
  \note[targetoffsetx=9.5cm, targetoffsety=4cm, angle=90, radius=3cm,
  width=14cm, linewidth=.2cm,
  roundedcorners=20, innersep=1cm]
  {\Large\centering$\displaystyle{n \choose k} = \dfrac{n!}{k!\cdot\left(n-k\right)!}$}
\end{columns}

\end{document}
